\documentclass[10pt,letterpaper,twoside]{article}

\usepackage[letterpaper,left=1in,right=1in,top=1.5in,bottom=1.5in]{geometry}

\title{Elliptic normal curves and the Brauer--Manin obstruction}
\author{Benjamin Antieau, Asher Auel, and Federico Scavia}
\date{\today}

%%%%%%%%%%%%%%%%%%            
%%% Dictionary %%%
%%%%%%%%%%%%%%%%%%
% idempotent complete
% nonconnective
% grouplike
% Brauer--Manin
% Severi--Brauer
% genus 1 curve
%%%%%%%%%%%%%%%%%%

\usepackage[pdfstartview=FitH,
pdfauthor={},
pdftitle={Elliptic normal curves},
colorlinks,
linkcolor=reference,
citecolor=citation,
urlcolor=e-mail,
backref]{hyperref}

% General includes
\usepackage{amsmath}
\usepackage{amscd}
\usepackage{amsbsy}
\usepackage{amssymb}
\usepackage{verbatim}
\usepackage{eufrak}
\usepackage{eucal}
\usepackage{microtype}
\usepackage{hyperref}
\usepackage{mathrsfs}
\usepackage{amsthm}
\usepackage{stmaryrd}
\pagestyle{headings}
\usepackage{xy}
\input xy
\xyoption{all}
\usepackage{tikz}
\usetikzlibrary{matrix,arrows}
\usepackage{tikz-cd}
\usepackage{dsfont}
\usepackage[T1]{fontenc}
\usepackage{enumitem}
\setlist{noitemsep}

% Headers
\usepackage{fancyhdr}
\pagestyle{fancy}
\renewcommand{\sectionmark}[1]{\markright{\thesection.\ #1}}
\fancyhead{}
\fancyhead[LO,R]{\bfseries\footnotesize\thepage}
\fancyhead[LE]{\bfseries\footnotesize\rightmark}
\fancyhead[RO]{\bfseries\footnotesize\rightmark}
\chead[]{}
\cfoot[]{}
\setlength{\headheight}{1cm}

% Prisms
\usepackage[bbgreekl]{mathbbol}
\usepackage{amsfonts}
\DeclareSymbolFontAlphabet{\mathbb}{AMSb} % to ensure \mathbb does not change
\DeclareSymbolFontAlphabet{\mathbbl}{bbold}
\newcommand{\Prism}{{\mathbbl{\Delta}}}
\newcommand{\Prismbar}{{\overline{\mathbbl{\Delta}}}}
\newcommand{\Inf}{{\mathbbl{\Pi}}}
\newcommand{\Infbar}{{\overline{\mathbbl{\Pi}}}}
\newcommand{\Strat}{{\mathbbl{\Sigma}}}
\newcommand{\Stratbar}{{\overline{\mathbbl{\Sigma}}}}
\usepackage{yhmath}
\newcommand{\prismhat}[1]{\widetriangle{#1}}
\newcommand{\tensorprism}{\prismhat{\otimes}}
\newcommand{\WCart}{\mathrm{WCart}}
\newcommand{\HT}{\mathrm{HT}}
\newcommand{\prism}{{ \mathbbl{\Delta}}}
\newcommand{\oprism}{\overline{\mathbbl{\Delta}}}
\newcommand{\pc}{\mathrm{pc}}
\newcommand{\wc}{\mathrm{wc}}

% Color definitions
\usepackage{color}
\definecolor{todo}{rgb}{1,0,0}
\definecolor{conditional}{rgb}{0,1,0}
\definecolor{e-mail}{rgb}{0,.40,.80}
\definecolor{reference}{rgb}{.20,.60,.22}
\definecolor{mrnumber}{rgb}{.80,.40,0}
\definecolor{citation}{rgb}{0,.40,.80}

% Font commands
\renewcommand{\bf}{\bfseries}
\renewcommand{\rm}{\mdseries}

% Some commenting features
\newcommand{\todo}[1]{\textcolor{todo}{#1}}
\newcommand{\conditional}[1]{\textcolor{conditional}{#1}}
\setlength{\marginparwidth}{1.2in}
\let\oldmarginpar\marginpar
\renewcommand\marginpar[1]{\-\oldmarginpar[\raggedleft\footnotesize #1]%
	{\raggedright\footnotesize #1}}
\renewcommand{\AA}[1]{\textcolor{blue}{#1}}
\newcommand{\FS}[1]{\textcolor{purple}{#1}}
\newcommand{\BA}[1]{\textcolor{olive}{#1}}

% Script letters
\newcommand{\Ascr}{\mathcal{A}}
\newcommand{\Bscr}{\mathcal{B}}
\newcommand{\Cscr}{\mathcal{C}}
\newcommand{\Dscr}{\mathcal{D}}
\newcommand{\Escr}{\mathcal{E}}
\newcommand{\Fscr}{\mathcal{F}}
\newcommand{\Gscr}{\mathcal{G}}
\newcommand{\Hscr}{\mathcal{H}}
\newcommand{\Iscr}{\mathcal{I}}
\newcommand{\Kscr}{\mathcal{K}}
\newcommand{\Lscr}{\mathcal{L}}
\newcommand{\Mscr}{\mathcal{M}}
\newcommand{\Nscr}{\mathcal{N}}
\newcommand{\Oscr}{\mathcal{O}}
\newcommand{\Pscr}{\mathcal{P}}
\newcommand{\Qscr}{\mathcal{Q}}
\newcommand{\Rscr}{\mathcal{R}}
\newcommand{\Sscr}{\mathcal{S}}
\newcommand{\Tscr}{\mathcal{T}}
\newcommand{\Vscr}{\mathcal{V}}
\newcommand{\Wscr}{\mathcal{W}}
\newcommand{\Xscr}{\mathcal{X}}

% Roman letters
\renewcommand{\a}{\mathrm{a}}
\newcommand{\A}{\mathrm{A}}
\renewcommand{\b}{\mathrm{b}}
\newcommand{\B}{\mathrm{B}}
\renewcommand{\c}{\mathrm{c}}
\newcommand{\C}{\mathrm{C}}
\renewcommand{\d}{\mathrm{d}}
\newcommand{\D}{\mathrm{D}}
\newcommand{\E}{\mathrm{E}}
\newcommand{\F}{\mathrm{F}}
\newcommand{\G}{\mathrm{G}}
\renewcommand{\H}{\mathrm{H}}
\newcommand{\h}{\mathrm{h}}
\newcommand{\I}{\mathrm{I}}
\newcommand{\K}{\mathrm{K}}
\renewcommand{\L}{\mathrm{L}}
\newcommand{\M}{\mathrm{M}}
\newcommand{\N}{\mathrm{N}}
\renewcommand{\P}{\mathrm{P}}
\newcommand{\R}{\mathrm{R}}
\renewcommand{\S}{\mathrm{S}}
\newcommand{\s}{\mathrm{s}}
\newcommand{\T}{\mathrm{T}}
\renewcommand{\t}{\mathrm{t}}
\newcommand{\U}{\mathrm{U}}
\renewcommand{\u}{\mathrm{u}}
\newcommand{\W}{\mathrm{W}}
\newcommand{\X}{\mathrm{X}}
\newcommand{\Z}{\mathrm{Z}}

% Bold letters
\renewcommand{\1}{\mathbf{1}}
\newcommand{\bA}{\mathbf{A}}
\newcommand{\bB}{\mathbf{B}}
\newcommand{\bC}{\mathbf{C}}
\newcommand{\bD}{\mathbf{D}}
\newcommand{\bE}{\mathbf{E}}
\newcommand{\bF}{\mathbf{F}}
\newcommand{\bG}{\mathbf{G}}
\newcommand{\bH}{\mathbf{H}}
\newcommand{\bI}{\mathbf{I}}
\newcommand{\bJ}{\mathbf{J}}
\newcommand{\bK}{\mathbf{K}}
\newcommand{\bL}{\mathbf{L}}
\newcommand{\bM}{\mathbf{M}}
\newcommand{\bN}{\mathbf{N}}
\newcommand{\bO}{\mathbf{O}}
\newcommand{\bP}{\mathbf{P}}
\newcommand{\bQ}{\mathbf{Q}}
\newcommand{\bR}{\mathbf{R}}
\newcommand{\bS}{\mathbf{S}}
\newcommand{\bT}{\mathbf{T}}
\newcommand{\bU}{\mathbf{U}}
\newcommand{\bV}{\mathbf{V}}
\newcommand{\bW}{\mathbf{W}}
\newcommand{\bX}{\mathbf{X}}
\newcommand{\bY}{\mathbf{Y}}
\newcommand{\bZ}{\mathbf{Z}}

\renewcommand{\mathds}{\mathbbl}

% Blackboard letters
\renewcommand{\AA}{\mathds{A}}
\newcommand{\BB}{\mathds{B}}
\newcommand{\CC}{\mathds{C}}
\newcommand{\EE}{\mathds{E}}
\newcommand{\FF}{\mathds{F}}
\newcommand{\GG}{\mathds{G}}
\newcommand{\MM}{\mathds{M}}
\newcommand{\NN}{\mathds{N}}
\newcommand{\PP}{\mathds{P}}
\newcommand{\QQ}{\mathds{Q}}
\newcommand{\RR}{\mathds{R}}
\renewcommand{\SS}{\mathds{S}}
\newcommand{\TT}{\mathds{T}}
\newcommand{\ZZ}{\mathds{Z}}

% Frakture letters
\newcommand{\Sfrak}{\mathfrak{S}}

% oo-categories
\newcommand{\Cat}{\Cscr\mathrm{at}}
\newcommand{\Catn}{\Cat_n}
\newcommand{\iCat}{\Cat_\infty}
\newcommand{\Fin}{\Fscr\mathrm{in}}
\newcommand{\Op}{\Oscr\mathrm{p}}
\newcommand{\Groth}{\Gscr\mathrm{roth}}
\newcommand{\Sp}{\Sscr\mathrm{p}}
\newcommand{\Spc}{\Sscr}

% Big categories
\newcommand{\Ab}{\Ascr\mathrm{b}}
\newcommand{\Topoi}{\Tscr\mathrm{op}}
\newcommand{\PrL}{\Pscr\mathrm{r}^{\mathrm{L}}}
\newcommand{\PrR}{\Pscr\mathrm{r}^{\mathrm{R}}}

% oo-category notation
\newcommand{\dual}{\mathrm{dual}}
\newcommand{\fd}{\mathrm{fd}}
\newcommand{\op}{\mathrm{op}}
\newcommand{\cofib}{\mathrm{cofib}}
\newcommand{\fib}{\mathrm{fib}}

% Additive and stable oo-category notation
\newcommand{\ex}{\mathrm{ex}}
\newcommand{\lex}{\mathrm{lex}}
\newcommand{\cn}{\mathrm{cn}}
\newcommand{\proj}{\mathrm{proj}}
\newcommand{\dg}{\mathrm{dg}}

% Algebraic categories
\newcommand{\GrMod}{\mathrm{GrMod}}
\newcommand{\Ch}{\mathrm{Ch}}
\newcommand{\Mod}{\mathrm{Mod}}
\newcommand{\coMod}{\mathrm{coMod}}
\newcommand{\RMod}{\mathrm{RMod}}
\newcommand{\QCoh}{\mathrm{QCoh}}
\newcommand{\qc}{\mathrm{qc}}
\newcommand{\Coh}{\mathrm{Coh}}
\newcommand{\Perf}{\mathrm{Perf}}
\newcommand{\Ind}{\mathrm{Ind}}
\newcommand{\Pro}{\mathrm{Pro}}
\newcommand{\Alg}{\mathrm{Alg}}
\newcommand{\CAlg}{\mathrm{CAlg}}
\newcommand{\ACAlg}{\mathrm{ACAlg}}
\newcommand{\DAlg}{\mathrm{DAlg}}
\newcommand{\coh}{\mathrm{coh}}
\newcommand{\DF}{\D\F}
\newcommand{\DGr}{\mathrm{DGr}}
\newcommand{\Shv}{\mathrm{Shv}}

% Gradings and filtrations
\newcommand{\Gr}{\mathrm{Gr}}
\newcommand{\gr}{\mathrm{gr}}
\newcommand{\ins}{\mathrm{ins}}
\newcommand{\fil}{\mathrm{fil}}

% Derived commutative rings
\newcommand{\LSym}{\mathrm{LSym}}
\newcommand{\Sym}{\mathrm{Sym}}

% Miscellaneous
\renewcommand{\Im}{\mathrm{Im}}
\newcommand{\inv}{\mathrm{inv}}
\newcommand{\nr}{\mathrm{nr}}
\newcommand{\Aut}{\mathrm{Aut}}
\newcommand{\Sha}{\mathrm{Sha}}
\newcommand{\WC}{\mathrm{WC}}
\newcommand{\an}{\mathrm{an}}
\renewcommand{\top}{\mathrm{top}}
\newcommand{\un}{\mathrm{un}}
\newcommand{\pro}{\mathrm{pro}}
\newcommand{\nil}{\mathrm{nil}}
\newcommand{\ft}{\mathrm{ft}}
\newcommand{\fin}{\mathrm{fin}}
\newcommand{\taut}{\mathrm{taut}}
\newcommand{\ccn}{\mathrm{ccn}}
\newcommand{\fsc}{\mathrm{fsc}}
\newcommand{\bFbar}{\overline{\bF}}
\newcommand{\Poly}{\mathrm{Poly}}
\newcommand{\Vect}{\mathrm{Vect}}
\newcommand{\perf}{\mathrm{perf}}
\newcommand{\CycSp}{\mathrm{CycSp}}
\newcommand{\syn}{\mathrm{syn}}
\newcommand{\ev}{\mathrm{ev}}
\newcommand{\Pairs}{\mathrm{Pairs}}
\newcommand{\tensorhat}{\widehat{\otimes}}
\newcommand{\tensor}{\otimes}
\newcommand{\init}{\mathrm{init}}
\newcommand{\der}{\mathrm{der}}
\newcommand{\site}{\mathrm{site}}
\newcommand{\dz}{{\mathrm{d}z}}
\newcommand{\can}{\mathrm{can}}
\newcommand{\xto}{\xrightarrow}
\newcommand{\heart}{\heartsuit}
\newcommand{\tors}{\mathrm{tors}}
\newcommand{\coev}{\mathrm{coev}}
\newcommand{\id}{\mathrm{id}}
\newcommand{\cof}{\mathrm{cof}}
\newcommand{\coker}{\mathrm{coker}}
\newcommand{\im}{\mathrm{im}}
\renewcommand{\geq}{\geqslant}
\renewcommand{\leq}{\leqslant}
\newcommand{\Ho}{\mathrm{Ho}}

% Structure sheaves
\newcommand{\sOX}{\Oscr_X}

% Homological functors
\DeclareMathOperator{\Tor}{Tor}
\DeclareMathOperator{\Ext}{Ext}
\newcommand{\HC}{\mathrm{HC}}
\newcommand{\THH}{\mathrm{THH}}
\newcommand{\MH}{\mathrm{MH}}
\newcommand{\MP}{\mathrm{MP}}
\newcommand{\HH}{\mathrm{HH}}
\newcommand{\HP}{\mathrm{HP}}
\newcommand{\TP}{\mathrm{TP}}
\newcommand{\TC}{\mathrm{TC}}
\newcommand{\TF}{\mathrm{TF}}
\newcommand{\TR}{\mathrm{TR}}
\newcommand{\KH}{\mathrm{KH}}
\newcommand{\KEnd}{\mathrm{KEnd}}
\newcommand{\KNil}{\mathrm{KNil}}
\newcommand{\KCoNil}{\mathrm{KCoNil}}
\newcommand{\KAut}{\mathrm{KAut}}
\newcommand{\KFor}{\mathrm{KFor}}
\newcommand{\GEnd}{\mathrm{GEnd}}
\newcommand{\GNil}{\mathrm{GNil}}
\newcommand{\GCoNil}{\mathrm{GCoNil}}
\newcommand{\GAut}{\mathrm{GAut}}
\newcommand{\GFor}{\mathrm{GFor}}

% de Rham cohomology
\newcommand{\BMS}{\mathrm{BMS}}
\newcommand{\RQRSPerfdscr}{\Rscr\Qscr\mathrm{RSPerfd}}
\newcommand{\QSynscr}{\Qscr\mathrm{Syn}}
\newcommand{\QSyn}{\mathrm{QSyn}}
\newcommand{\qSyn}{\mathrm{qSyn}}
\newcommand{\QRSPerfdscr}{\Qscr\mathrm{RSPerfd}}
\newcommand{\dR}{\mathrm{dR}}
\newcommand{\conj}{\mathrm{conj}}
\newcommand{\dRhat}{\widehat{\dR}}
\newcommand{\Ocp}{\Oscr_{\bC_p}}
\newcommand{\Oc}{\Oscr_{\bC}}
\newcommand{\Prismhat}{\widehat{\Prism}}
\newcommand{\GM}{\mathrm{GM}}
\newcommand{\DMIC}{\mathrm{DMIC}}
\newcommand{\crys}{\mathrm{crys}}
\newcommand{\stack}{\mathrm{stack}}
\renewcommand{\inf}{\mathrm{inf}}

% Cohomology theories
\newcommand{\ku}{\mathrm{ku}}
\newcommand{\KU}{\mathrm{KU}}
\newcommand{\ko}{\mathrm{ko}}
\newcommand{\KO}{\mathrm{KO}}
\newcommand{\TMF}{\mathrm{TMF}}
\newcommand{\MU}{\mathrm{MU}}

% Mapping objects
\newcommand{\Map}{\mathrm{Map}}
\newcommand{\bMap}{\mathbf{Map}}
\newcommand{\Hom}{\mathrm{Hom}}
\newcommand{\Fun}{\mathrm{Fun}}
\newcommand{\End}{\mathrm{End}}

% Sheaves
\newcommand{\bBr}{\mathbf{Br}}
\newcommand{\bBBr}{\mathbf{BBr}}
\newcommand{\bPic}{\mathbf{Pic}}
\newcommand{\bBPic}{\mathbf{BPic}}

% Algebraic and compact Lie groups
\newcommand{\Gm}{\bG_{m}}
\newcommand{\Ga}{\bG_{a}}
\newcommand{\GL}{\mathbf{GL}}
\newcommand{\PGL}{\mathbf{PGL}}
\newcommand{\SL}{\mathbf{SL}}

% Classifying spaces
\newcommand{\BGm}{\mathbf{BG}_m}
\newcommand{\BGL}{\mathbf{BGL}}
\newcommand{\BPGL}{\mathbf{BPGL}}

% Brauer groups
\newcommand{\Pic}{\mathrm{Pic}}
\newcommand{\Br}{\mathrm{Br}}
\newcommand{\per}{\mathrm{per}}
\newcommand{\ind}{\mathrm{ind}}
\renewcommand{\deg}{\mathrm{deg}}

% Limits and colimits
\DeclareMathOperator*{\holim}{holim}
\DeclareMathOperator*{\colim}{colim}
\DeclareMathOperator*{\hocolim}{hocolim}
\DeclareMathOperator*{\Tot}{Tot}
\newcommand{\LEq}{\mathrm{LEq}}

% Topologies on schemes
\newcommand{\fppf}{\mathrm{fppf}}
\newcommand{\et}{\mathrm{\acute{e}t}}
\newcommand{\proet}{\mathrm{pro\acute{e}t}}
\newcommand{\Zar}{\mathrm{Zar}}
\newcommand{\Nis}{\mathrm{Nis}}

% Schemes
\newcommand{\Stk}{\mathrm{Stk}}
\DeclareMathOperator{\Spec}{Spec}
\DeclareMathOperator{\FSpec}{FSpec}
\DeclareMathOperator{\GrSpec}{GrSpec}
\DeclareMathOperator{\Spf}{Spf}
\DeclareMathOperator{\SSpf}{\SS pf}
\DeclareMathOperator{\Proj}{Proj}
\newcommand{\red}{\mathrm{red}}

% Special arrows
\newcommand{\triplerightarrows}{%
	\tikz[minimum height=0ex]
	\path[->]
	node (a)            {}
	node (b) at (1em,0) {}
	(a.north)  edge (b.north)
	(a.center) edge (b.center)
	(a.south)  edge (b.south);%
}
\newcommand{\lrlarrows}{%
	\tikz[minimum height=0ex]
	\path[->]
	node (a)            {}
	node (b) at (1em,0) {}
	(b.north)  edge (a.north)
	(a.center) edge (b.center)
	(b.south)  edge (a.south);%
}
\newcommand{\we}{\simeq}
\newcommand{\iso}{\cong}
\newcommand{\rcof}{\rightarrowtail}
\newcommand{\rfib}{\twoheadrightarrow}
\newcommand{\lcof}{\leftarrowtail}
\newcommand{\lfib}{\twoheadleftarrow}

% Theorems
\theoremstyle{plain}
\newtheorem{theorem}{Theorem}[section]
\newtheorem*{theorem*}{Theorem}
\newtheorem{lemma}[theorem]{Lemma}
\newtheorem{scholium}[theorem]{Scholium}
\newtheorem{proposition}[theorem]{Proposition}
\newtheorem{conditionalprop}[theorem]{Conditional Proposition}
\newtheorem{conditionaltheorem}[theorem]{Conditional Theorem}
\newtheorem{conjecture}[theorem]{Conjecture}
\newtheorem{corollary}[theorem]{Corollary}
\newtheorem*{corollary*}{Corollary}
\newtheorem{ideal}[theorem]{Ideal Theorem}
\newtheorem{problem}[theorem]{Problem}
\newtheorem{statement}[theorem]{Statement}

\theoremstyle{plain}
\newtheorem{maintheorem}{Theorem}
\newtheorem{mainmetatheorem}[maintheorem]{Meta Theorem}
\newtheorem{mainconjecture}[maintheorem]{Conjecture}
\newtheorem{maincorollary}[maintheorem]{Corollary}

\theoremstyle{definition}
\newtheorem{mainquestion}[maintheorem]{Question}
\newtheorem{mainexample}[maintheorem]{Example}
\newtheorem{mainremark}[maintheorem]{Remark}

\newtheoremstyle{named}{}{}{\itshape}{}{\bfseries}{.}{.5em}{#1 \thmnote{#3}}
\theoremstyle{named}
\newtheorem*{namedtheorem}{Theorem}
\newtheorem*{namedconjecture}{Conjecture}

\theoremstyle{definition}
\newtheorem{definition}[theorem]{Definition}
\newtheorem{warning}[theorem]{Warning}
\newtheorem{variant}[theorem]{Variant}
\newtheorem{aphorism}[theorem]{Aphorism}
\newtheorem{notation}[theorem]{Notation}
\newtheorem{assumption}[theorem]{Assumption}
\newtheorem{hypothesis}[theorem]{Hypothesis}
\newtheorem{example}[theorem]{Example}
\newtheorem*{example*}{Example}
\newtheorem{examples}[theorem]{Examples}
\newtheorem{question}[theorem]{Question}
\newtheorem*{question*}{Question}
\newtheorem{construction}[theorem]{Construction}
\newtheorem{calculation}[theorem]{Calculation}
\newtheorem{remark}[theorem]{Remark}
\newtheorem{convention}[theorem]{Convention}
\newtheorem{exercise}[theorem]{Exercise}
\newtheorem{future}[theorem]{Future work}
\newtheorem{claim}[theorem]{Claim}

\begin{document}

\maketitle
\tableofcontents



\section{Introduction}

Let $k$ be a field, let $\Bar{k}$ be a separable closure of $k$, let $N\geq 1$ be an integer, and
let $A$ be a central simple algebra of degree $N$ over $k$. We write $\bP(A)$ for the Severi--Brauer
variety of $A$. The central simple algebra $A$ defines a class $\alpha\in\Br(k)$. It is an open
problem of Clark and Saltman to determine whether $\alpha$ (or equivalently $A$) is split by a genus $1$ curve, which is
to say whether there exists a smooth proper geometrically connected genus $1$ curve $C$ defined over
$k$ such that the image of $\alpha$ in $\Br(k(C))$ vanishes.

The genus $1$ curve splitting problem, and variants of it, have been considered
in~\cite{dejong-ho,ho-lieblich,antieau-auel,saltman-genus,huybrechts-mattei}. The problem of determining
which Brauer classes are split by a given genus $1$ curve has been studied
by~\cite{ciperiani-krashen}, who give algorithms over number fields.

Here is a brief summary of what is known; for more details and examples, see~\cite{antieau-auel}.
\begin{enumerate}
    \item[(a)] Over an arbitrary field, geometric arguments using the linear algebra of vector
        bundles on Severi--Brauer varieties show that central simple algebras of
        degree at most $5$ are split by genus $1$ curves~\cite{dejong-ho}.
    \item[(b)] Over an arbitrary field, every Brauer class is split by a torsor for the Jacobian of
        a curve or a product of two curves~\cite{ho-lieblich}.
    \item[(c)] Over a local field, every Brauer class is split by a genus $1$ curve~\cite[Ex.~2.4]{antieau-auel}.
    \item[(d)] Over a number field, every central simple algebra of degrees $N=2,\ldots,10$ or $N=12$ is split
        by a genus $1$ curve~\cite{antieau-auel} (with the assumption that $k$ contains a primitive $8$th root of unity
        if $N=8$, an element of the form $\zeta_9+\zeta_9^{-1}$, where $\zeta_9$ is a primitive
        $9$th root of unity, if $N=9$, and a primitive $4$th root of unity if $N=12$).
    \item[(e)] If there is an elliptic curve $E$ over $k$ with full level $N$ structure
        $E[N]\iso\bZ/N\times\mu_N$, then every {\em cyclic} central simple algebra of degree $N$ is
        split by a genus $1$ curve~\cite{antieau-auel}.
\end{enumerate}

Geometrically, a central simple algebra $A$ is split by a genus $1$ curve if and only if there is a genus $1$ curve $C$
defined over $k$ and a $k$-linear morphism $C\rightarrow\bP(A)$. The geometry of this map is
unspecified.

By definition, if $A$ is a central simple algebra of degree $N$, then an elliptic normal curve in $\bP(A)$ is a
smooth projective genus $1$ curve $C\subset \bP(A)$ such that $C_{\Bar{k}}\subset
\bP(A_{\Bar{k}})\simeq \bP_{\Bar{k}}^{N-1}$ has degree $N$ and is not contained in any hyperplane of
$\bP_{\Bar{k}}^{N-1}$.
% (This definition is independent of the choice of the isomorphism
% $\bP(A_{\Bar{k}})\simeq \bP_{\Bar{k}}^{n-1}$.)
Equivalently, the embedding
$C_{\Bar{k}}\subseteq\bP(A_{\Bar{k}})\iso\bP^{N-1}$ is defined by a full linear system of degree $N$
on $C_{\Bar{k}}$. Note that despite the name, if $[A]\neq 0$ in $\Br(k)$, then $\bP(A)$ has no
rational points. Hence, if $C\subseteq\bP(A)$ is an elliptic normal curve, then $C$ has no rational points either, so it is a
genus $1$ curve only, not an elliptic curve.

\begin{theorem}\label{mainthm}
    Let $k$ be a number field, $p$ be a prime number, and $A$ be a central simple algebra of degree
    $p$ over $k$. Then there exists an elliptic normal curve $C\subset \bP(A)$ over $k$. In
    particular, $C$ splits $A$.
\end{theorem}

In fact, our method works for central simple algebras of more general square-free integer degrees
$N\geq 2$, which we
call $\GL_2$-general, meaning that if $p\;|\;N$, then the order of a $p$-Sylow subgroup of
$\GL_2(\bZ/N)$ is $p$.\footnote{\BA{Comparing a list of $\GL_2$-general numbers generated by a short
{\ttfamily SAGE} script and the {\ttfamily OEIS} turns up integer sequence {\ttfamily A350342},
which is the sequence of integers $N$ such that every group of order $N^2$ is abelian. I conjecture
that these sequences are the same and I checked roughly up to $N=9997$.}} Among the numbers up to $100$, besides the primes, the following integers are
$\GL_2$-general:
$$35,\quad 65,\quad 77,\quad 85.$$
Thus, for example, for $N=35,65,77,85$, every twisted form of $\bP^{N-1}$ over a number field $k$ contains an
elliptic normal curve. Among numbers up to $10\,000$ there are $1229$ primes and $2510$ numbers which
are $\GL_2$-general.

Our proof of Theorem~\ref{mainthm} is to rephrase the problem as one of finding rational points on
certain Hilbert scheme constructions. Namely, if $E$ is an elliptic curve, we let $V_E^A$ be the scheme of elliptic normal curves
$C$ in $\bP(A)$ together with a fixed isomorphism $\Pic_{C/k}^0\iso E$.
This is a smooth scheme which is an $\Aut_{E/k}$-torsor over $V_{j(E)}^A$, the closed subscheme of the
Hilbert scheme of elliptic normal curves in $\bP(A)$ whose Jacobians have $j$-invariant $j(E)$.
Here, $\Aut_{E/k}$ is the group scheme of automorphisms of the elliptic curve $E$, i.e.,
automorphisms of the scheme $E$ which preserve the basepoint $0\in E$.

We prove that for countably many $j$-invariants $j_0$ there are judicious choices of $E$ with
$j(E)=j_0$ such that the scheme $V_E^A$ has $k_v$-points for all places $v$. Then, we show that for
all but finitely many of these choices there is no Brauer--Manin obstruction to the existence of
rational points by showing that the unramified Brauer group of $V_E^A$ is the image
of $\Br(k)\rightarrow\Br(V_E^A)$.\footnote{This part of our argument is the only one where we need
to fix $N$ to be $\GL_2$-general.} As $V_E^A$ is geometrically a homogeneous space for $\SL_N$ with
stabilizer a Heisenberg group of order $N^3$, this is already enough to conclude that for these $E$ the scheme $V_E^A$ has a
$0$-cycle of degree $1$ by a theorem of \BA{who?}. We conclude by using \BA{forthcoming} work of Demeio,
which shows that for spaces which are geometrically homogeneous under actions of semisimple simply
connected algebraic groups with solvable stabilizers, like the Heisenberg groups of order $N^3$ in
$\SL_N$, the Brauer--Manin obstruction to rational points is the only one.
This is a special case of the open Colliot-Th\'el\`ene conjecture which states that for geometrically
rationally connected varieties the Brauer--Manin obstruction is the only obstruction to the
existence of rational points.

\BA{TODO: remark that in fact one gets density of rational points and hence each $V_E^A(k)$ is in
fact countably infinite.}



\paragraph{Ackowledgments.}
We thank Danny Krashen, Eoin Mackall, and David Saltman for conversations related to the topic of the
present paper. Thanks also go to Julian Demeio for sharing with us his results on the Brauer--Manin
obstruction before they were released.

The first author was supported by NSF grants DMS-2102010 and DMS-2152235, by Simons Fellowship
666565, and by the Simons Collaboration on Perfection (Award MP-SCMPS-00001529-11).





\section{Hilbert schemes of elliptic normal curves}

\begin{definition}
    Fix $N\geq 3$. An elliptic normal curve in $\bP^{N-1}$ is the image of an embedding
    $C\hookrightarrow\bP^{N-1}$ defined by a complete linear system of degree $N$ on a genus $1$
    curve $C$ and a choice of a projective basis for the space of sections.
\end{definition}

\begin{question}
    What is the correct definition for $N=2$? What is the scheme of double covers of $\bP^1$
    branched at four points?
\end{question}

The elliptic normal curves define a smooth locus inside the Hilbert scheme of
$\bP^{N-1}$.

\begin{construction}
    Let $V(N)$ denote the Hilbert scheme of smooth elliptic normal curves in $\bP^{N-1}$. Taking the
    Jacobian of the universal curve yields a surjective morphism $V(N)\rightarrow\Mscr_{1,1}$. Fix
    an elliptic curve $E$ with $j$-invariant $j(E)$ and define $V_{j(E)}$ and
    $V_{E_0}$ as pullbacks in the commutative diagram
    $$\xymatrix{
        V_{E}\ar[r]\ar[d]&V_{j(E)}\ar[r]\ar[d]&V(N)\ar[d]\\
        \ast\ar[r]^{E_0}&\B\Aut_{E}\ar[r]\ar[d]&\Mscr_{1,1}\ar[d]^j\\
        &\ast\ar[r]^{j(E)}&\bA^1.
    }$$ 
    of pullback squares. The map $V_{j(E)}\rightarrow V(N)$ is a closed embedding, while
    $V(E)\rightarrow V_{j(E)}$ is the total space of an $\Aut_{E}$-torsor.
    The scheme $V_{j(E)}$ represents the moduli problem of elliptic normal curves
    $C\hookrightarrow\bP^{N-1}$ whose Jacobian
    has $j$-invariant $j(E)$, while $V_E$ represents ths moduli problem of elliptic normal curves
    $C\hookrightarrow\bP^{N-1}$ with a fixed isomorphism $\Pic^0_{C/k}\iso E$.
\end{construction}

\begin{question}
    Vakil and Zinger study a compactification of $V(N)$
    in~\cite{vakil-zinger,vakil-zinger-announce}. How does their compactification interact with the
    Jacobian fibration above?
\end{question}

\begin{proposition}
    Fix an elliptic curve $E$ and an integer $N\geq 1$.
    The action of $\PGL_N$ on $\bP^{N-1}$ induces an action on $V(N)$ which preserves $V_{j(E)}$ and
    lifts to an action on $V_E$. This action makes $V_E$ into a non-empty homogenous space for $\PGL_N$ with
    stabilizers isomorphic to $E[N]$.
\end{proposition}

\begin{proof}
    As $\PGL_N$ acts on $\bP^{N-1}$, it acts on the Hilbert scheme of $\bP^{N-1}$ and this action
    preserves the locus $V(N)$ of elliptic normal curves and also the locus $V_{j(E)}$.
    Given $g\in\PGL_N(k)$ and a rational point $x$ of $V_E$ corresponding to $C\hookrightarrow\bP^{N-1}$
    together with an isomorphism $f\colon\Pic^0_{C/k}\rightarrow E$, the point $g\cdot x$
    corresponds to $C\hookrightarrow\bP^{N-1}\xrightarrow{g}\bP^{N-1}$. As $g$ induces an isomorphism
    $C\rightarrow g(C)$, there is an induced isomorphism
    $\Pic^0_{g(C)/k}\xrightarrow{g^*}\Pic^0_{C/k}\xrightarrow{f}$, which shows that the action of
    $\PGL_N$ lifts to $V_E$.

    To complete the proof, we work over an algebraically closed field $k$ and suppose that $x$ and $y$
    are two points of $V_E(k)$ corresponding to two degree $N$ line bundles $\Lscr_x$ and
    $\Lscr_y$ on $E$ and choices of bases $\{x_1,\ldots,x_N\}$ and $\{y_1,\ldots,y_N\}$ of
    $\H^0(E,\Lscr_x)$ and $\H^0(E,\Lscr_y)$, respectively. Let $t$ be the point of $E$
    corresponding to $\Lscr_x^{-1}\otimes\Lscr_y\in\Pic^0_{E/k}(k)\iso E(k)$. Then,
    $t^*\Lscr_x\iso\Lscr_y$. Note that translation by a point of $E$ does not change the point of
    $V_E$ since it is only the closed subscheme of $\bP^{N-1}$ which matters. Thus, we can assume
    that $\Lscr_x=\Lscr_y$ and that the points of $V_E$ differ only by the choice of a projective basis of
    $\H^0(\Lscr_y)$. Any two choices differ by an element of $\PGL_N$, which proves that $V_E$ is a
    homogeneous space for $\PGL_N$. It remains to identify the stabilizer. Let $e$ be the identity
    element of $E$ and let $\Lscr=\Oscr(Ne)$. Choose a projective basis for $\H^0(E,\Lscr)$. It is a \BA{standard}
    fact that the subgroup of translations of $E$ that extend to automorphisms of $\bP^{N-1}$ is
    exactly $E[N]$; in particular, there is a closed embedding $E[N]\rightarrow\PGL_N$. If
    $g\in\PGL_N(k)$ stabilizes this point, then it acts as an isomorphism from $E$ to itself.
    However, it must be a translation since if it were an outer automorphism it would change the
    identification of $\Pic^0_{E/k}$ with $E$. By the standard fact, it follows that $g\in E[N](k)$.
    Conversely, given $g\in E[N](k)$, we have for a degree $N$ line bundle $\Lscr$ that
    $g^*\Lscr\iso\Lscr$. The corresponding element of $\PGL_N(k)$ is induced from our choice of
    projective basis and the isomorphisms
    $\H^0(E,\Lscr)\xrightarrow{g^*}\H^0(E,g^*\Lscr)\iso\H^0(E,\Lscr)$. \BA{Finish!}
\end{proof}

\begin{construction}
    Given a point $x\in V_E(k)$, the induced morphism $\PGL_N\rightarrow V_E$ is
    $\PGL_N$-equivariant. It follows that given a principal $\PGL_N$-torsor $P^A$, corresponding to
    an Azumaya algebra $A$ of degree $N$, there is an $E[N]$-torsor $$P^A\rightarrow V_E^A,$$
    where $V_E^A$ is the twisted form of $V_E$ induced by $A$. One can construct $V_E^A$ directly as
    the moduli space of elliptic normal curves $C\hookrightarrow\bP(A)$ in the Severi--Brauer
    variety $\bP(A)$ equipped with a fixed isomorphism $\Pic^0_{C/k}\iso E$.
    The cover $P^A\rightarrow V_E^A$ on the other hand depends on the rational point $x\in V_E(k)$.
    \BA{Or, we can directly construct the quotient $P^A/E[N]$.}
\end{construction}


\begin{proposition}
    Let $k$ be a field, $E$ be an elliptic curve, $A$ be a central simple algebra of degree $n\geq 1$ over $k$. There exists a $\mathrm{SL}(A)$-homogeneous space $V_E^A$ such that:
    \begin{itemize}
    \item if $\Bar{v}\in V_E^A(\Bar{k})$, the geometric stabilizer $H_{\Bar{v}}\subset \mathrm{SL}(A_{\Bar{k}})\simeq \mathrm{SL}_n$ is conjugate to $\mathrm{He}_n$. 
    \item for all fields extensions $K/k$, $V_{E/A}(K)\neq\emptyset$ if and only if there exist an elliptic normal curve $C\subset \bP(A_K)$ over $K$ and an isomorphism of elliptic curves $J(C)\simeq E_K$.
    \end{itemize}
\end{proposition}


	\section{Unramified Brauer group calculations}

\subsection{Preliminaries on unramified Brauer groups of homogeneous spaces}

\begin{definition}
	Let $k$ be a field of characteristic zero and let $V$ be a smooth geometrically integral scheme over $k$. The unramified Brauer group $\Br_{\nr}(V)\subset \Br(V)$ of $V$ is defined as the image of $\Br(V^c)$ in $\Br(V)$, where $V^c$ is a smooth projective compactification of $V$. (The definition of $\Br_{\nr}(V)$ does not depend on the choice of $V^c$.) We
	let $\Br_1(V)$ be the kernel of the restriction map $\Br(V)\to \Br(V_{\Bar{k}})$. We let $\Br_{1,\nr}(V)=\Br_\nr(V)\cap\Br_1(V)$ be the subgroup of $\Br(V)$ of geometrically trivial unramified Brauer classes. This group contains as a
	subgroup the group $\Br_0(V)=\im(\Br(k)\rightarrow\Br(V))$.
\end{definition}

Suppose that $k$ is a number field, let $\Omega_k$ be the set of places of $k$ and, for every $v \in \Omega_k$, write $\text{inv}_v: Br(k_v) \to \mathbb{Q}/\mathbb{Z}$ for the Hasse invariant. We have the Brauer--Manin pairing
\[
\Br_\nr(V) \times \prod_{v \in \Omega_k} V(k_v) \to \mathbb{Q}/\mathbb{Z}, \qquad (\beta, (x_v)) \mapsto \sum_{v \in \Omega_k} \text{inv}_v \, \beta(x_v).
\]
Note that $\sum_{v \in \Omega_k} \text{inv}_v \, \beta(x_v)$ is finite because $\beta$ is
unramified. The subset of $\prod_{v \in \Omega_k} V(k_v)$ consisting of the $(x_v)_{v\in \Omega_k}$
that are orthogonal to $\Br_\nr(V)$ with respect to the Brauer--Manin pairing is called the
Brauer--Manin set. The standard exact sequence
$$0\rightarrow\Br(k)\rightarrow\bigoplus_v\Br(k_v)\xrightarrow{\sum\inv_v}\bQ/\bZ\rightarrow 0$$ of class field theory implies that the image of $V(k)$ in $\prod_{v \in \Omega_k} V(k_v)$ is contained in the Brauer--Manin set.

The following theorem is due to Demeio \cite{}.

\begin{theorem}[Demeio]
	Let $k$ be a number field, let $V$ be a homogeneous space of a semi-simple and simply connected
	linear group $G$ with finite solvable geometric stabilizers, and let $V^c$ be a smooth
	compactification of $V$. Then $V^c(k)$ is dense in the Brauer–Manin set of $V^c$ with respect to
	the product of the $v$-adic topologies. In particular, if the Brauer--Manin set of $V$ is
	non-empty, then $V$ has a $k$-rational point.
\end{theorem}



\subsection{The unramified Brauer group of a homogeneous space}
	Let $k$ be a field, let $G$ be a simply connected semisimple $k$-group, let $V$ be a homogeneous space under $G$, let $\Bar{v}\in V(\Bar{k})$ be a geometric point of $V$, and let $H_{\Bar{v}}$ be the geometric stabilizer of $\Bar{v}$. We suppose that $H_{\Bar{v}}$ is finite.
	
	Let $G_{\Bar{v}} \subseteq G(\Bar{k}) \rtimes \Gamma_k$ be the subgroup consisting of those pairs $(g, \sigma)$ such that $g\sigma(\Bar{v}) = \Bar{v}$. We have a short exact sequence of abstract groups
	\[
	1 \to H_{\Bar{v}} \to G_{\Bar{v}} \to \Gamma_k \to 1.
	\]
	This short exact sequence induces an outer action of $\Gamma_k$ on $H_{\Bar{v}}$, that is, a continuous group homomorphism $\Gamma_k\to \mathrm{Out}(H_{\Bar{v}})$. 
	
	We write $\mu_{\infty}\subset \Bar{k}^\times$ for the $\Gamma_k$-module given by the roots of unity in $\Bar{k}^\times$. We let $H^{\mathrm{ab}}_{\Bar{v}}$ be the abelianization of  $H^{\mathrm{ab}}$, and we let $\hat{H}^{\mathrm{ab}}_{\Bar{v}}= \mathrm{Hom}(H^{\mathrm{ab}}_{\Bar{v}}, \mu_{\infty})$ be the Cartier dual of $H^{\mathrm{ab}}_{\Bar{v}}$. By \cite{}, we have an isomorphism of $\Gamma_k$-modules $\text{Pic}(V_{\Bar{k}})\simeq \hat{H}^{\mathrm{ab}}_{\Bar{v}}$. As a consequence, the Hochschild–Serre spectral sequence provides an exact sequence:
	\[
	\text{Br}(k) \to \text{Br}_1(V) \to H^1(k, \hat{H}^{\mathrm{ab}}_{\Bar{v}}) \xrightarrow{\delta} H^3(k, \bar{k}^*).
	\]
	The map $\delta$ is trivial if $V$ admits a zero-cycle of degree $1$ or if $k$ is a number field; see \cite{}. If this is the case, we obtain a short exact sequence
	\[0\to \Br_0(V)\to \Br_1(V)\to H^1(k, \hat{H}^{\mathrm{ab}}_{\Bar{v}})\to 0.\]
	Let $e$ denote the outer exponent of $H_{\Bar{v}}$, and fix a finite Galois subextension $k\subset L\subset \Bar{k}$, satisfying the following three conditions:
	\begin{enumerate}
		\item $L$ contains all $e$-th roots of unity;
		\item $V(L) \neq \emptyset$;
		\item the outer action of $\Gamma_k$ on $H_{\Bar{v}}$ factors through $\mathrm{Gal}(L/k)$.
	\end{enumerate}
	We have the inflation-restriction exact sequence \[H^1(\mathrm{Gal}(L/k),\hat{H}^{ab}_{\Bar{v}})\xrightarrow{\mathrm{Inf}}H^1(k,\hat{H}^{ab}_{\Bar{v}})\to H^1(L,\hat{H}^{ab}_{\Bar{v}}).\]	
	We write $\chi:\Gamma_k\to \hat{\bZ}^\times$ for the cyclotomic character. Let $\sigma \in \mathrm{Gal}(L/k)$ and $u \in H_{\Bar{v}}$ be such that $\sigma(u)$ is conjugate to $u^{\chi(\sigma)}$. Then the image $\Bar{u}$ of $u$ in $H^{ab}_{\Bar{v}}$ satisfies $\sigma(\Bar{u}) = \chi(\sigma)\Bar{u}$, and is hence $\sigma$-invariant when considered as an element of $\text{Hom}(\hat{H}^{ab}_{\Bar{v}}, \Bar{\mu}_\infty)$, where $\Bar{\mu}_\infty$ denotes the group of roots of unity in $\Bar{k}^\times$ equipped with the trivial Galois action. In particular, if we let $\mathbb{Z}$ act on $\text{Hom}(\hat{H}^{ab}_{\Bar{v}}, \Bar{\mu}_\infty)$ via $\sigma$, then we may view $u$ as an element of $H^0(\mathbb{Z}, \text{Hom}(\hat{H}^{ab}_{\Bar{v}}, \Bar{\mu}_\infty))$. We obtain a composite map:
	\[
	H^1(\mathrm{Gal}(L/k), \hat{H}^{ab}_{\Bar{v}}) \xrightarrow{\sigma^*} H^1(\mathbb{Z}, \hat{\overline{H}}_{ab}) \xrightarrow{x \mapsto u \cup x} H^1(\mathbb{Z}, \mu_\infty) = \mu_\infty,
	\]
	where $\sigma^*$ is the pull-back map along $\mathbb{Z} \rightarrow \mathrm{Gal}(L/k)$ given by $1 \mapsto \sigma$.
	
	The following theorem is due to Harpaz--Wittenberg \cite[Proposition 3.3]{}. It builds upon formulae of Harari, Demarche, and Lucchini Arteche.
	
	
	\begin{theorem}\label{brauer-homogeneous}
		There is an equality
		\begin{align*}
			\Br_{1,\nr}(V)/\Br_0(V) = \{\beta \in H^1(\mathrm{Gal}(L/k), \hat{H}^{\mathrm{ab}}_{\Bar{v}}) \,|\,& \delta(\beta) = 0 \text{ and } u \cup \sigma^*\beta = 1 \text{ in $\mu_\infty$ $\forall u \in \hat{H}_{\Bar{v}}$ and $\forall \sigma \in \mathrm{Gal}(L/k)$}\\
			&\text{such that $\sigma(u)$ is conjugate to $u^{\chi(\sigma)}$ in $\hat{H}_{\Bar{v}}$}\},
		\end{align*}
	\end{theorem}
	
	We conclude this section with the following simple observation.
	
	\begin{lemma}\label{injective-res}
		Let $k'$ be a Galois field extension of $k$ such that the restriction map $H^1(k,\hat{H}^{\mathrm{ab}}_{\Bar{v}})\to H^1(k',\hat{H}^{\mathrm{ab}}_{\Bar{v}})$ is injective. Then the pullback homomorphism 
		\[\Br_{1,\nr}(V)/\Br_0(V)\to \Br_{1,\nr}(V_{k'})/\Br_0(V_{k'})\]
		is injective.
	\end{lemma}
	
	\begin{proof}
		This follows from a diagram chase in the commutative diagram of exact sequences of abelian groups
		\[
		\begin{tikzcd}
			0\arrow[r] &  \Br_0(V) \arrow[r] \arrow[d] & \Br_1(V) \arrow[r] \arrow[d] & H^1(k,\hat{H}^{\mathrm{ab}}_{\Bar{v}})\arrow[d,hook] \\
			0\arrow[r] &  \Br_0(V_{k'}) \arrow[r] & \Br_1(V_{k'}) \arrow[r] & H^1(k',\hat{H}^{\mathrm{ab}}_{\Bar{v}})
		\end{tikzcd}
		\]
		where the rows come from the Hochschild-Serre spectral sequence and the vertical arrows are restriction maps. 
	\end{proof}
	
	
\subsection{Elliptic normal curves}
	
	\begin{proposition}\label{sylow-case}
		Let $E$ be an elliptic curve over a field $k$ of characteristic zero and let $N$ be an integer. Let $V_E$ be the
		Hilbert scheme of elliptic normal curves $C$ in $\bP^{N-1}_k$
		with a fixed isomorphism $\Pic_{C/k}^0\iso E$. 
		If the image of the Galois group of $k$ in $\GL_2(\bZ/N)$ associated to $E[N]$
		is conjugate to the unipotent group $$U_2=\left\{\begin{pmatrix}1&*\\0&1\end{pmatrix}\colon
		*\in\bZ/N\right\},$$ then the pullback map $$\Br(k)\rightarrow\Br_{1,\nr}(V_E)$$
		is an isomorphism.
	\end{proposition}
	
	\begin{proof}
	Since $V_E$ has a rational point, corresponding to $E$ embedded into $\bP^{N-1}_k$ via a line bundle $\Lscr$ of degree $N$ and a choice of basis of $\H^0(E,\Lscr)$, the map
		$\Br(k)\rightarrow\Br_{1,\nr}(V_E)$ is injective with image $\Br_0(V)$. 
		
		In order to prove that $\Br(k)\rightarrow\Br_{1,\nr}(V_E)$ is surjective, we use the criterion of
		Harpaz--Wittenberg~\cite{harpaz-wittenberg-massey}. The Hochschild-Serre spectral sequence yields an exact sequence
		\[\Br(k)\to \Br(V_E)\to H^1(k,\Pic(\Bar{V}))\xrightarrow{\delta}H^3(k,\Bar{k}^\times).\]
		Here we have used the fact that $\Bar{k}[V_E]^\times=\Bar{k}^\times$, by Rosenlicht's Lemma \cite{}. Moreover, since $V_E(k)\neq \emptyset$, the map $\delta$ is equal to zero. Finally, by \cite{}, we have a $\Gamma_k$-equivariant isomorphism $\Pic(\Bar{V}_E)\simeq H^{\mathrm{ab}}_N$, and hence
		\[\Pic(\Bar{V}_E)\simeq E[N](\Bar{k}).\]
		
		It follows that the map $\Br(V_E)\to H^1(k,\Pic(\Bar{V}))$ identifies $\Br(V_E)/\Br_0(V_E)$ with a subgroup of $H^1(k,E[N])$. We now apply \cite[Proposition 3.3]{harpaz-wittenberg-massey} to obtain the following. An element $\beta\in H^1(k,E[N])$ belongs to $\Br(V_E)/\Br_0(V_E)$ if and only if for all $\sigma\in \Gamma_k$ and $u\in H_N$ such that $\sigma(u)$ is conjugate to $u^{\chi(\sigma)}$, the pairing
		\[E[N](\Bar{k})^{\langle\sigma\rangle}\times E[N](\Bar{k})/(\sigma-1)\to \mu_N(\Bar{k})\]
		induced by the Weil pairing sends $(\beta_{\sigma},\Bar{u})$ to $1$. 
	\end{proof}
	
	\begin{remark}
		The assumption on the Galois representation implies that $E[N]$ sits inside a maximally non-split exact sequence
		$$0\rightarrow\bZ/N\rightarrow E[N]\rightarrow\bZ/N\rightarrow 0.$$
	\end{remark}
	
	
	
	\begin{proposition}\label{restrict}
	Let $k$ be a field, let $N\geq 1$ be an integer invertible in $k$, let $A\in \Br(k)$ be a Brauer class, let $k'/k$ be a finite field extension, and let $E$ be an elliptic curve over $k$. Suppose that either
	\begin{enumerate}
		\item $k'/k$ is Galois, $A_{k'}=0$ and $E[N](k')=\{0\}$, or
		\item $[k':k]$ is prime to $N$.
	\end{enumerate}
	Then the restriction map $\Br_1(V_E)/\Br_0(V_E)\to \Br_1((V_E)_{k'})/\Br_0((V_E)_{k'})$ is injective. In particular, if $\Br_{1,nr}((V_E)_{k'})=\Br_0((V_E)_{k'})$, then $\Br_{1,nr}(V_E)=\Br_0(V_E)$.
	\end{proposition}
	
	\begin{proof}
		In view of Lemma \ref{injective-res}, it suffices to show that the assumptions imply the injectivity of $H^1(k,H^{\mathrm{ab}}_{\Bar{v}})\to H^1(k',H^{\mathrm{ab}}_{\Bar{v}})$. 
		
		Suppose that (1) holds. By the inflation-restriction sequence it suffices to show that $H^1(k'/k,(H^{\mathrm{ab}}_{\Bar{v}})^{\Gamma_{k'}})=0$. Since $A_{k'}$ is split, we have an isomorphism of $\Gamma_{k'}$-modules $H^{\mathrm{ab}}_{\Bar{v}}\simeq E[N](\Bar{k})$, and hence  $(H^{\mathrm{ab}}_{\Bar{v}})^{\Gamma_{k'}}\simeq E[N](k')$. By assumption $E[N](k')=\{0\}$, and hence $H^1(k'/k,(H^{\mathrm{ab}}_{\Bar{v}})^{\Gamma_{k'}})=0$.
		
		If (2) holds, the conclusion follows from a restriction-corestriction argument.
	\end{proof}

\section{Local points everywhere via Tate curves}

\subsection{Local fields}

Suppose that $K$ is a local field with uniformizer $\varpi$ and residue field $k\iso\bF_{\ell^f}$ and that $E$ is an elliptic curve over $K$. Then, Tate and Milne 
showed that there is a non-degenerate continuous pairing $E(K)\times\WC(E/K)\rightarrow\bQ/\bZ$
where $E(K)$ is given the topology arising from the local field and $\WC(E/K)$ is given the discrete
topology.

Assume that $p$ is a prime invertible in $k$ and so large that no elliptic curve over $k$ admits
a $p$-torsion point (for example via the Hasse bound $|p-\ell^f-1|>2\sqrt{\ell^f}$).
In order for $\WC(E/K)$ to admit a $p$-torsion point, by the Tate pairing, we must thus have that
$E$ has split multiplicative
reduction and that $p$ divides $-v_{\varpi}(j(E))=v_\varpi(\Delta(E))$.

We can always arrange this by letting $E$ be the Tate curve
$E=K^\times/(\varpi^p)^{\bZ}$, which has split multiplicative reduction
and $j$-invariant with $\varpi$-adic valuation $-p$.

When $p$ is not large, the situation for general $E$ is less clear, but it is still the case that the Tate curve
$E=K^\times/(\varpi^p)$ will have a torsor splitting the $p$-torsion Brauer class (at least if $p$
is invertible in $K$).


\subsection{Elliptic curves with split multiplicative reduction at specified primes}
	
Now, let $K=\bQ$, $p$ a prime, and $\ell_1,\ldots,\ell_r$ other primes.
	
\begin{construction}\label{const:multiplicative}
	Let $E$ be an elliptic curve over $\bQ$ with $j$-invariant
	$$j(E)=\frac{1}{\ell_1^p\cdots\ell_r^p}.$$
	At each completion, $E_{\bQ_{\ell_i}}$ is a quadratic twist of a Tate curve
	(see~\cite[Thm.~C.14.1]{silverman}). Let $\chi_i$ denote the corresponding $i$th quadratic character. By the
	Grunwald--Wang theorem, there is a character $\chi$ over $\bQ$ such that $\chi$ completes toneach $\chi_i$. It follows that the quadratic twist $E_\chi$, which has the same $j$-invariant,	has split multiplicative reduction at each $\ell_i$.
\end{construction}
	
\begin{proposition}
	Let $k$ be a global field and let $A$ be a central simple algebra over $k$, ramified at finite primes $\pi_1,\ldots,\pi_r$. For any $j$-invariant $j_0$ satisfying $v_{\pi_i}(j)=-N$ for $1\leq i\leq r$ there is an elliptic curve $E$ defined over $k$ of	$j$-invariant $j(E)=j_0$ such that $V_E^A(k_v)$ is non-empty for all places $v$.
	\end{proposition}
	
\begin{proof}
	At finite places, this follows from the construction. At infinite real places $v$, $E(k_v)$ admits a
	$k_v$-rational $2$-torsion point and hence the restriction of $\alpha$ to $k_v$ is split by an $E$-torsor corresponding to a choice of dual of this $2$-torsion point under the Tate pairing.
\end{proof}
	
	
	
	\section{Proof of Theorem \ref{mainthm}}
	
	\begin{proof}
		Let $V_E^A$ be the Hilbert scheme of elliptic normal curves $C$ in $\bP(A)$ with a fixed isomorphism $\Pic_{C/k}^0\iso E$. By ???, the collection of elliptic curves $E$ over $k$ such that $V_E^A(k_v)\neq \emptyset$ for every place $v$ of $k$ is infinite. By ???, the collection of elliptic curves $E$ over $k$ such that the group $E[p](k')$ is non-trivial is finite. Therefore, there exists an elliptic curve $E$ over $k$ such that $E[p](k')$ is trivial and $V_E^A(k_v)\neq \emptyset$ for every place $v$ of $k$.
		
		We now show that $\Br_{1,\nr}(V_E^A)=\Br_0(V_E^A)$. Since $k$ is a number field, there exists a degree-$N$ cyclic Galois extension $k'/k$ such that $A_{k'}=0$ in $\Br(k')$. Let $\Delta\subset \Gamma_{k'}$ be the inverse image of a $p$-Sylow subgroup of $\mathrm{Out}(H_{\Bar{v}})$, and let $k'\subset k''\subset\Bar{k}$ be the finite extension of $k'$ fixed by $\Delta$. By Proposition \ref{sylow-case}, we have $\Br_{1,\nr}((V_E^A)_{k''})=\Br_0((V_E^A)_{k''})$. By Proposition \ref{restrict}, we first deduce that $\Br_{1,\nr}((V_E^A)_{k'})=\Br_0((V_E^A)_{k'})$, and then that $\Br_{1,\nr}(V_E^A)=\Br_0(V_E^A)$. We conclude by ???.
	\end{proof}
	
	
	\section{Future directions}
	
	\paragraph{Are there ever unramified Brauer classes?} We can prove that under certain assumptions on
	$E$ that the split form $V_E$ has no non-constant unramified classes. Are there ever non-constant
	unramified classes? Either in the split or the non-split case? One way to show this would be to
	prove that $\Pic_{V^c_E/k}$ is a constant lattice, so that $\H^1(\Spec k,\Pic_{V^c_E/k})=0$. Now,
	I'm not sure how to prove this, but Vakil and Zinger study
	in~\cite{vakil-zinger,vakil-zinger-announce} compactificiations of the Hilbert scheme $V$ of elliptic normal
	curves. This is obtained at blowing up at certain naturally defined types of degenerations. These
	subspaces are defined over $\bZ$ (or perhaps $\bZ[\tfrac{1}{N}]$) and their arguments give
	compactifications of $V_{j(E)}^\alpha$ as well. I wonder, since everything is so stable, if this is
	enough to make a geometric argument about $\Pic_{V^c_E/k}$.
	
	Relatedly, there is a surjective map $\Pic_{V^c_E/k}\rightarrow\Pic_{V_E/k}$. I would love to know
	that $\Pic_{V^c_E/k}$ were torsion free so that the kernel of this map were also torsion free. In
	fact, is it the case that for any smooth projective {\em rational} variety one hac $\Pic_{X/k}$ is
	torsion free?
	
	The idea would then be to argue that as $E$ varies over the moduli of elliptic curves one gets some
	lattice of fixed rank. Then, one should argue that $\pi_1\bA^1=0$ to conclude that in fact there is
	no variation and make some further arguments to conclude that $\Pic_{V^c_E/k}$ is a constant
	lattice, from which it would follow that the unramified classes are always constant. Is this
	feasible at all?
	
	\paragraph{Can we just compute Brauer--Manin directly?} We have such control over the local elements
	of $V_E^k$ that it tempting to show directly that there are elements in the Brauer--Manin fixed
	locus. The issue is that even in the split case given $\alpha\in\Br(V_E)$ and $x\in V_E(k_v)$ it is
	hard to understand what $x^*\alpha$ is.
	Nevertheless, this should be entirely controlled by the geometry of the situation. If possible, we
	could presumably generalize Theorem~\ref{mainthm} to all $N$.
	
	\subsection{Evaluating the Brauer--Manin obstruction}
	
	
	\begin{lemma}
		Let $G$ be a simply connected semisimple algebraic group over a field $k$ \BA{of characteristic
			$0$}. Let $V$ be a homogenous space for $G$ of
		the form $G/H$ where $H\subseteq G$ is a finite (solvable?) subgroup and let $f\colon
		V\rightarrow\Spec k$ be the structure morphism. We have
		\begin{align*}
			\R^0f_*\bG_{m,V}&\iso\Gm,\\
			\R^1f_*\bG_{m,V}&\iso\widehat{H},\\
			\R^2f_*\bG_{m,V}&=0,
		\end{align*}
		where $\widehat{H}$ is the Cartier dual of $H$, which agrees with the Cartier dual of the
		abelianization of $H$.
	\end{lemma}
	
	\begin{proof}
		Let $g\colon G\rightarrow\Spec k$ be the structure morphism for $G$.
		Then, $\R^0g_*\bG_{m,G}\iso\Gm$ by
		\BA{Rosenlicht's lemma}, $\R^1g_*\bG_{m,G}=0$ by simple connectedness, and $\R^2g_*\bG_{m,G}=0$
		by a result of Iversen~\cite{iversen} since $G$ is simply connected. Let $h\colon G\rightarrow
		V$ be the quotient map. We have that $\R^0h_*\bG_{m,G}$ a torus of rank the order of $H$, while
		$\R^1h_*\bG_{m,G}=\R^2h_*\bG_{m,G}=0$. 
		Using the spectral sequence $$\E_2^{s,t}=\R^s f_*(\R^t h_*\bG_{m,G})\Rightarrow\R^{s+t}
		g_*\bG_{m,G},$$
		As the natural pullback map $\R^0f_*\bG_{m,V}\rightarrow\R^0g_*\bG_{m,G}$ is
		injective, it follows that the Rosenlicht's lemma holds for $V$ as well. The claim about
		$\R^1f_*\bG_{m,V}$ is \BA{standard}.
	\end{proof}
	
	
	
	
\section{A new approach}

In~\cite{lawson-wuthrich}, the authors prove that $\H^1(G,E[N](\overline{k}))$ typically vanishes where $G$ is
the image of Galois in $\GL_2(\bZ/N)$. Using their ideas, we show that we can always choose $E$ so
that $$\frac{\Br_{1,\nr}(V_E^A)}{\Br_0(V_E^A)}=0.$$ Suppose that $[A]=\theta\cup u$ where
$\theta\in\H^1(\Spec k,\bZ/N)$ and $u\in\H^1(\Spec k,\mu_N)$.

For infinitely many $j$-invariants there is an elliptic curve $E$ such that for all places $v$ there is
an $E$-torsor over $k_v$ which splits $[A_v]\in\Br(k_v)$. We show that there are also many whose
$N$-torsion satisfies certain conditions.

\begin{lemma}
    Suppose $N>1$ is an odd prime power, $k$ is a number field, and $A$ is a central simple algebra of degree $N$.
    There are infinitely many $j$-invariants such that there exists an elliptic curve
    $E$ defined over $k$ of that $j$-invariant satisfying the following conditions:
    \begin{enumerate}
        \item[{\em (a)}] for all places $v$ there is an $E$-torsor over $k_v$ which splits
            $[A_v]\in\Br(k_v)$;
        \item[{\em (b)}] $\Br_{1,\nr}(V_E^A)/\Br_0(V_E^A)=0$.
    \end{enumerate}
\end{lemma}

\begin{proof}
    Infinitely $j$-invariants $j_s$ with elliptic curves $E_s$ realizing (a) have been established above.
%     For infinitely many $s$, one has $E_s[N](k)=0$.
%     If $N$ is sufficiently large, this follows from
%     Falting's theorem because the genus of the corresponding modular curve $X_1(N)$ is at least $2$.
%     In that case, (b) also holds for infinitely many $s$, by applying Falting's theorem over $k_\theta$.
% 
%     \BA{We assume now that $N=p^i$ is an odd prime power.}
    Possibly after a quadratic twist of $E_s$ away from
    the ramification primes of $[A]$, we can assume that $$\begin{pmatrix}-1&0\\0&-1\end{pmatrix}$$
    is in the image of Galois. This is a non-trivial homothety of $E[N]$. Consider the morphism
    $$\Gamma_k\rightarrow\GL_2(\bZ/N)\times\bZ/N$$ obtained from the product of the Galois
    representation attached to $E[N]$ and the character $\theta$. Let $H$ denote the kernel.
    Then, $\Gamma_k/H$ fits into an exact sequence
    $$0\rightarrow\bZ/N'\rightarrow\Gamma_k/H\rightarrow\Im\rightarrow 0,$$ where $\Im$ is the
    image of the Galois inside $\GL_2(\bZ/N)$ and $N'$ is some divisor of $N$ corresponding to the
    order of the image of $\theta$ in $\H^1(k(E[N]),\bZ/N)$. The group $\Gamma_k/H$ satisfies the hypotheses
    needed to apply~\cite[Prop.~3.3]{harpaz-wittenberg-massey}. In particular,
    $$\Br_{1,\nr}(V_E^A)/\Br_0(V_E^A)\hookrightarrow\H^1(\Gamma_k/H,E[N](\overline{k}))$$
    (since $E$ has full level $N$ structure over $\overline{k}^H$ by construction).
    Note also that the residual action of $\bZ/N'$ on $E[N](\overline{k})$ is trivial. The
    inflation-restriction
    exact sequence takes the form $$0\rightarrow
    \H^1(\Im,E[N](\overline{k})^{\bZ/N'})\rightarrow\H^1(\Gamma_k/H,E[N](\overline{k}))\rightarrow\H^1(\bZ/N',E[N](\overline{k}))^{\Im}.$$
    By the remark above, $E[N](\overline{k})^{\bZ/N'}\iso E[N](\overline{k})$. As $\Im$ contains a
    non-trivial homothety and $N$ is a prime power, it follows from~\cite[Lem.~3]{lawson-wuthrich} that the
    left-hand term vanishes. On the other hand, since the $\bZ/N'$-action is trivial,
    $$\H^1(\bZ/N',E[N](\overline{k}))^{\Im})\subseteq E[N'](\overline{k})^{\Im}\subseteq
    E[N](\overline{k})^{\Im}=0,$$
    since $\Im$ contains a non-trivial homothety.
\end{proof}

\begin{remark}
    I expect the same result works for all odd $N>1$, but I do not have time to check. For $N$ even,
    a bit more work has to be done to handle the $2$-torsion, but it should be OK.
\end{remark}
	
	
	
	
	
	
	
	
	
	
	\small
	\bibliographystyle{amsplain}
	\bibliography{enc}
	
\end{document}
